\documentclass[answers]{exam}
\makeindex

\usepackage{amsmath, amsfonts, amssymb, amstext, amscd, amsthm, makeidx, graphicx, hyperref, url, enumerate}
\newtheorem{theorem}{Theorem}
\allowdisplaybreaks

\begin{document}

\begin{center}
{\Large CS 156a - Problem Set 3} \\
\medskip
Marco Yang \\
\medskip
2237027
\bigskip
\end{center}

\begin{questions}

\section*{Generalization Error}

\question
The modified Hoeffding Inequality provides a way to characterize the 
generalization error with a probabilistic bound:

\[
\mathbb{P}\left[ |E_{in}(g) - E_{out}(g)| > \epsilon \right] \leq 2M 
e^{-2\epsilon^2 N}
\]

for any $\epsilon > 0$. If we set $\epsilon = 0.05$ and want the 
probability bound $2M e^{-2\epsilon^2 N}$ to be at most 0.03, what is 
the least number of examples $N$ (among the given choices) needed for 
the case $M = 1$?

\begin{choices}
    \choice 500
    \choice 1000
    \choice 1500
    \choice 2000
    \choice More examples are needed
\end{choices}

\begin{solution}
B. 1000

\[
2(1)e^{-2(0.05)^2N} \le 0.03 \implies N \ge 839.94
.\]  
\end{solution}

\question
Repeat for the case $M = 10$.

\begin{choices}
    \choice 500
    \choice 1000
    \choice 1500
    \choice 2000
    \choice More examples are needed
\end{choices}

\begin{solution}
C. 1500

\[
2(100)e^{-2(0.05)^2N} \le 0.03 \implies N \ge 1300.46
.\] 
\end{solution}

\question
Repeat for the case $M = 100$.

\begin{choices}
    \choice 500
    \choice 1000
    \choice 1500
    \choice 2000
    \choice More examples are needed
\end{choices}

\begin{solution}
D. 2000

\[
2(100)e^{-2(0.05)^2N} \le 0.03 \implies N \ge 1760.98
.\] 
\end{solution}

\section*{Break Point}

\question
As shown in class, the (smallest) break point for the Perceptron Model 
in the two-dimensional case ($\mathbb{R}^2$) is 4 points. What is the 
smallest break point for the Perceptron Model in $\mathbb{R}^3$? (i.e., 
instead of the hypothesis set consisting of separating lines, it 
consists of separating planes.)

\begin{choices}
    \choice 4
    \choice 5
    \choice 6
    \choice 7
    \choice 8
\end{choices}

\begin{solution}
A. 5.

Consider a plane of points with one label and two points on each side
of the plane with the other label. It is impossible to draw a plane
that shatters these 5 points properly.
\end{solution}

\section*{Growth Function}

\question
Which of the following are possible formulas for a growth function 
$m_\mathcal{H}(N)$:

\begin{itemize}
    \item[i)] $1 + N$
    \item[ii)] $1 + N + \binom{N}{2}$
    \item[iii)] $\sum_{i=1}^{\lfloor \sqrt{N} \rfloor} \binom{N}{i}$
    \item[iv)] $2^{\lfloor N/2 \rfloor}$
    \item[v)] $2^N$
\end{itemize}

\begin{choices}
    \choice i, v
    \choice i, ii, v
    \choice i, iv, v
    \choice i, ii, iii, v
    \choice i, ii, iii, iv, v
\end{choices}

\begin{solution}
E. i, ii, iii, iv, v

We can simply have our hypothesis be to predict from a set of $x \le 2r^{N}$ number 
of dichotomies.
\end{solution}

\section*{Fun with Intervals}

\question
Consider the “2-intervals” learning model, where $h: \mathbb{R} \to 
\{-1,+1\}$ and $h(x) = +1$ if the point is within either of two 
arbitrarily chosen intervals and $-1$ otherwise. What is the (smallest) 
break point for this hypothesis set?

\begin{choices}
    \choice 3
    \choice 4
    \choice 5
    \choice 6
    \choice 7
\end{choices}

\begin{solution}
C. 5.

For 4 or less points, you can assign one interval to each point, no matter
their label. However, with 5 points, you can assign alternating colors
such that the two intervals cannot contain 3 points in 3 different intervals.
\end{solution}

\question
Which of the following is the growth function $m_H(N)$ for the 
“2-intervals” hypothesis set?

\begin{choices}
    \choice $\binom{N+1}{4}$
    \choice $\binom{N+1}{2} + 1$
    \choice $\binom{N+1}{4} + \binom{N+1}{2} + 1$
    \choice $\binom{N+1}{4} + \binom{N+1}{3} + \binom{N+1}{2} + 
    \binom{N+1}{1} + 1$
    \choice None of the above
\end{choices}

\begin{solution}
C. $\binom{N}{4} + \binom{N}{2} + 1$.

The intervals could either be unique, overlap to form 1 interval,
or cover no points. Thus, there are $\binom{N}{4}$ ways to create
2 unique intervals, $\binom{N}{2}$ ways to create one unique interval,
and $1$ way to cover no points (no intervals).
your interval
\end{solution}

\question
Now, consider the general case: the “$M$-intervals” learning model. 
Again $h : \mathbb{R} \to \{-1,+1\}$, where $h(x) = +1$ if the point 
falls inside any of $M$ arbitrarily chosen intervals, otherwise $h(x) = 
-1$. What is the (smallest) break point of this hypothesis set?

\begin{choices}
    \choice $M$
    \choice $M + 1$
    \choice $M^2$
    \choice $2M + 1$
    \choice $2M - 1$
\end{choices}

\begin{solution}
D. $2M + 1$.

With $M$ intervals, you can always cover $2M$ points because the worst case
scenario is alternating labels for consecutive points, for which you need
exactly $M$ intervals. However, if you have $2M + 1$ points and alternate 
labels for each point starting with $h(x_1) = +1$, you can no longer assign
enough intervals to cover each of the $M+1$ points with label +1.

\end{solution}

\section*{Convex Sets: The Triangle}

\question
Consider the “triangle” learning model, where $h : \mathbb{R}^2 \to 
\{-1,+1\}$ and $h(x) = +1$ if $x$ lies within an arbitrarily chosen 
triangle in the plane and $-1$ otherwise. Which is the largest number 
of points in $\mathbb{R}^2$ (among the given choices) that can be 
shattered by this hypothesis set?

\begin{choices}
    \choice 1
    \choice 3
    \choice 5
    \choice 7
    \choice 9
\end{choices}

\begin{solution}
B. 3

With 3 points, you could create a small triangle for zero points,
a small triangle around single point for a single point, an infinitely thin 
triangle for two points, and a big triangle for three points.

For more than 3 points, you can create 3 points with label 1 and place a point 
with label -1 inside the triangle formed by these 3 points. This is impossible 
to shatter.
\end{solution}

\section*{Non-Convex Sets: Concentric Circles}

\question
Compute the growth function $m_H(N)$ for the learning model made up of 
two concentric circles around the origin in $\mathbb{R}^2$. 
Specifically, $H$ contains the functions which are +1 for
\[
a^2 \leq x_1^2 + x_2^2 \leq b^2
\]
and $-1$ otherwise, where $a$ and $b$ are the model parameters. The 
growth function is:

\begin{choices}
    \choice $N + 1$
    \choice $\binom{N+1}{2} + 1$
    \choice $\binom{N+1}{3} + 1$
    \choice $2N^2 + 1$
    \choice None of the above
\end{choices}

\begin{solution}
B. $\binom{N+1}{2} + 1$

We can $N+1$ choices for how many points to include each circle radii, from
0 points to $N$ points. There are $\binom{N+1}{2}$ ways to do so. However,
did not consider the cases in which we either choose both radii to contain 0
points or both radii to contain all points, which are the same dichotomory and
thus the $+1$ and not $+2$.
\end{solution}

\end{questions}
\end{document}
